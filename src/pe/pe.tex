%-----//
%
% DESCRIÇÃO
%
%   - Plano de Ensino
%
% AUTOR
%
%   - Eduardo Heredia (contato@eduardoheredia.pro.br)
%
% HISTÓRICO
%
%   - 01/02/2015: Versão inicial
%
%-----\\

\documentclass[a4paper]{article}

\usepackage[T1]{fontenc}
\usepackage[utf8]{inputenc}
\usepackage[brazil]{babel}
\usepackage{url}
\usepackage{indentfirst}
\usepackage{color}
\usepackage{listings}
\usepackage{booktabs}
\usepackage{multirow}
\usepackage{longtable}
\usepackage[portuguese,ruled,linesnumbered]{algorithm2e}

\newcommand{\versao}{versão \today}

\pdfinfo{
    /Author   (Prof. Eduardo Heredia <eduardo.heredia@sp.senac.br>)
    /Title    (Plano de Ensino Resumido)
    /Subject  (Plano de Ensino para a disciplina Algoritmos e Linguagem de Programação)
    /Keywords (Plano de Ensino; Algoritmo; Linguagem de Programação)
}

\title{Algoritmos e Linguagem de Programação}
\author{Prof. Eduardo Heredia\footnote{eduardo.heredia@sp.senac.br}}
\date{Plano de Ensino Resumido\\\tiny{\versao}}

\begin{document}

    \maketitle

    \section{Ementa}

        Apresenta uma visão geral do processo de programação e os
        fundamentos das linguagens de programação: valores e variáveis,
        comandos básicos, atribuições, comandos condicionais, comandos
        de iteração, comandos de seleção e funções com o objetivo de
        desenvolver algoritmos e estratégias de solução, representação,
        simulação e documentação no contexto de problemas de análise
        automatizada de dados enfrentados por profissionais em
        prática.

    \section{Objetivos}

        Introduzir conceitos básicos de programação e desenvolver o
        raciocínio lógico necessário para a resolução automatizada de
        problemas. Apresentar na prática os principais fundamentos do
        paradigma procedural através de programas escritos em uma
        linguagem específica.

    \section{Metodologia de Ensino}

        \begin{description}

            \item[Aulas Teóricas] apresentação dos fundamentos teóricos
                com análise e discussão dos assuntos apresentados
                utilizando-se recursos multimídia e textos de apoio;

            \item[Aulas Práticas] ilustração dos fundamentos teóricos
                através da experimentação prática;

            \item[Atividades Discentes Orientadas (ADO)] atividades
               realizadas extra-classe pelos alunos com o intuito de
               reforçar conceitos desenvolvidos em sala de aula, como
               por exemplo: resolução de listas de exercícios, e
               desenvolvimento de código computacional sem auxíliio direto
               do professor.

        \end{description}

    \section{Plano de Aulas}

        \begin{longtable}{lll}

            \toprule

            Aula & Conteúdo programático & Recurso previsto\\

            \toprule
            \toprule

            \multirow{3}{*}{Aula 00} & Apresentação da disciplina                      & Lousa e projetor \\ \cmidrule{2-3}
                                     & Apresentação das ferramentas de desenvolvimento & Laboratório      \\ \cmidrule{2-3}
                                     & ADO: tutorial code.org                                             \\ \midrule

            \multirow{3}{*}{Aula 01} & Conceitos básicos para programação              & Lousa e projetor \\ \cmidrule{2-3}
                                     & Oi, mundo!                                      & Laboratório      \\ \cmidrule{2-3}
                                     & ADO: Lista de exercícios de fixação                                \\ \midrule

            \multirow{3}{*}{Aula 02} & Execução Condicional I                          & Lousa e projetor \\ \cmidrule{2-3}
                                     & IF/ELSE                                         & Laboratório      \\ \cmidrule{2-3}
                                     & ADO: Lista de exercícios de fixação                                \\ \midrule

            \multirow{3}{*}{Aula 03} & Execução Condicional II                         & Lousa e projetor \\ \cmidrule{2-3}
                                     & SWITCH/CASE                                     & Laboratório      \\ \cmidrule{2-3}
                                     & ADO: Lista de exercícios de fixação                                \\ \midrule

            \multirow{3}{*}{Aula 04} & Execução Iterativa I                            & Lousa e projetor \\ \cmidrule{2-3}
                                     & FOR                                             & Laboratório      \\ \cmidrule{2-3}
                                     & ADO: Lista de exercícios de fixação                                \\ \midrule

            \multirow{3}{*}{Aula 05} & Execução Iterativa II                           & Lousa e projetor \\ \cmidrule{2-3}
                                     & WHILE/DO\ldots{}WHILE                           & Laboratório      \\ \cmidrule{2-3}
                                     & ADO: Lista de exercícios de fixação                                \\ \midrule

            Aula 06                  & Avaliação Escrita I                             & Sala de aula     \\ \midrule

            \multirow{3}{*}{Aula 07} & Introdução a vetores                            & Lousa e projetor \\ \cmidrule{2-3}
                                     & vetor[]                                         & Laboratório      \\ \cmidrule{2-3}
                                     & ADO: Lista de exercícios de fixação                                \\ \midrule

            \multirow{3}{*}{Aula 08} & Introdução a strings                            & Lousa e projetor \\ \cmidrule{2-3}
                                     & ``texto''                                       & Laboratório      \\ \cmidrule{2-3}
                                     & ADO: Lista de exercícios de fixação                                \\ \midrule

            \multirow{3}{*}{Aula 09} & Criação de funções I                            & Lousa e projetor \\ \cmidrule{2-3}
                                     & retorno funcao(parâmetros)                      & Laboratório      \\ \cmidrule{2-3}
                                     & ADO: Lista de exercícios de fixação                                \\ \midrule

            \multirow{3}{*}{Aula 0A} & Criação de funções II                           & Lousa e projetor \\ \cmidrule{2-3}
                                     & retorno funcao(parâmetros)                      & Laboratório      \\ \cmidrule{2-3}
                                     & ADO: Lista de exercícios de fixação                                \\ \midrule

            \multirow{3}{*}{Aula 0B} & Introdução a matrizes                           & Lousa e projetor \\ \cmidrule{2-3}
                                     & matriz[][]                                      & Laboratório      \\ \cmidrule{2-3}
                                     & ADO: Lista de exercícios de fixação                                \\ \midrule

            Aula 0C                  & Avaliação Escrita II                            & Sala de aula     \\ \midrule

            \multirow{3}{*}{Aula 0D} & Introdução a ponteiros I                        & Lousa e projetor \\ \cmidrule{2-3}
                                     & *p; p->                                         & Laboratório      \\ \cmidrule{2-3}
                                     & ADO: Lista de exercícios de fixação                                \\ \midrule

            \multirow{3}{*}{Aula 0E} & Introdução a ponteiros II                       & Lousa e projetor \\ \cmidrule{2-3}
                                     & retorno (* f)(parâmetros)                       & Laboratório      \\ \cmidrule{2-3}
                                     & ADO: Lista de exercícios de fixação                                \\ \midrule

            Aula 0F                  & Semana de Projetos Interativos                  &                  \\ \midrule

            \multirow{3}{*}{Aula 10} & Estudo dirigido                                 & Lousa e projetor \\ \cmidrule{2-3}
                                     & Resolução de problema real do BEAS              & Laboratório      \\ \cmidrule{2-3}
                                     & ADO: Lista de exercícios de fixação                                \\ \midrule

            Aula 11                  & Avaliação Escrita III (AE3)                     & Sala de aula     \\ \bottomrule

        \end{longtable}

        \subsection{Acompanhamento de Estudos}

            \begin{itemize}
                \item Todo código computacional desenvolvido pelo aluno
                      deverá ser compartilhado via GitHub;
                \item Toda ADO desenvolvida pelo aluno deverá ser compartilhada
                      via GitHub; 
                \item As devolutivas das avaliações escritas serão feitas
                      sempre na semana subsequente a sua realização.
            \end{itemize}

    \begin{thebibliography}{9}

        \bibitem{KR1} KERNIGHAN, B.W.; RITCHIE, D. C: A linguagem de
            programação. Campus, 1989.

        \bibitem{KR2} KERNIGHAN, B.W.; PIKE, R: A prática da programação.
            Campus, 2000.

        \bibitem{Xavier} XAVIER, G.F.C. Lógica de programação. São Paulo:
            Senac São Paulo, 2004.

	\end{thebibliography}

\end{document}
